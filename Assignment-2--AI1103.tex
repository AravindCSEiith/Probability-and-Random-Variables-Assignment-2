\documentclass[11pt,a4paper,twocolumn]{article}
\usepackage[utf8]{inputenc}
\usepackage{setspace}
\usepackage{graphicx}
\usepackage{hyperref}
\usepackage{amsmath}
\title{Assignment-2-Probability And Random Variables}
\author{Name: Aravinda Kumar Reddy Thippareddy\\
Roll.No.: CS20BTECH11053 }
\date{\today}

\begin{document}


\maketitle
\textbf{Problem Statement:} Probability density function $p(x)$ of random variable x is as shown below. The value of ${\alpha}$ is
\begin{figure}[h]
\centering
 \includegraphics[scale=0.2]{Screenshot_20210318-151757.png}
\end{figure}
\\\textbf{Solution:}
We know that, $$\int_{-\infty}^{+\infty}P(x).dx=1$$
\begin{align}
    \int_{-\infty}^{+\infty}P(x){\times}dx&=\int_{-\infty}^{\alpha}0\times{dx}+\int_{\alpha}^{{\alpha}+c}P(x){\times}dx\\
    &+\int_{\alpha+c}^{+\infty}0\times{dx}\\
    1&=0+\int_{\alpha}^{\alpha+c}P(x)\times{dx}+0\\
    1&=\frac{1}{2}\times{(\alpha+c-\alpha)}\times{\alpha}\\
    \frac{2}{c}&=\alpha\\
    {\alpha}&=\frac{2}{c}
\end{align}
\fbox{\begin{minipage}{20em}
Therefore, the value of $\alpha=\frac{2}{c}$.
\end{minipage}}


\end{document}
